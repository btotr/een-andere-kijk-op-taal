
%%%%%%%%%%%%%%%%%%%%%%%%%%%%%%%%%%%%%%%%%%%%%%%
%%% CONTENT %%%%%%%%%%%%%%%%%%%%%%%%%%%%%%%%%%%
%%%%%%%%%%%%%%%%%%%%%%%%%%%%%%%%%%%%%%%%%%%%%%%


\title{een andere kijk op taal}
Colin Meerveld

\starttext
\startfrontmatter
\intro{voorwoord}
Taal is zeg maar echt niet mijn ding. Bestempeld als dyslect had ik mij daar bij neergelegd. Ik
was ellenlang bezig om mijn ideeën te formuleren — vaak werd ik alsnog niet begrepen. Tot
mijn 36ste was dit het geval.
Ik denk anders en daar ben ik mij inmiddels bewust van. De vraag is of taalgevoel te leren is.
De traditionele manier van leren heeft mij weinig goeds gebracht dus moest het anders. De
manier hoe ik complexe vraagstukken op los is door ze te vereenvoudigen tot de essentie.
Precies dit heb ik gedaan met taal.
Op de lagere school wordt taal onderwezen voor mensen met taalgevoel, de meerderheid. Zij
leren spellingregels, rijtjes met uitzonderingen, ezelsbruggetjes en weten dit toe te passen.
Wanneer je dit taalgevoel niet hebt zou je een stap terug moeten zetten. Je moet boven de
materie staan en een taalgevoel kunstmatig creëren.
Ik ben een dyslecticus maar geen orthopedagoog. Deze uitleg over taal is de manier die voor
mij werkt. Ik ben ook zeker geen taalpurist en zal sommige taalkundige betekenissen losjes
gebruiken. Het doel is niet om Grammatica aan te leren maar een taalgevoel te creëren. Met dit
taalgevoel zou jij je beter moeten kunnen uitdrukken, weinig spelfouten maken en andermans
ideeën beter begrijpen.
Je zult dus ook geen volledig naslag vinden om correct spellingsregels toe te passen.
Integendeel, ik ben van mening dat een taalgevoel ontwikkelen velen malen belangrijker is.
Over het algemeen weet ik de spellingregels maar had geen idee wanneer ik het moest
toepassen. Iedereen kan het ezelsbruggetje van ‘t kofschip leren maar als je niet weet dat een
zin in verschillende tijden — en zelfs micro tijden — kan staan kun je het niet toepassen.
\stopfrontmatter

\completecontent

\chapter{De basis}
Je gebruikt taal om te communiceren. Je wilt bijvoorbeeld een idee uitleggen of een
mededeling maken. Je kunt zeggen dat je informatie uitgewisseld. Informatie uitwisselen doe je
door relaties (verbanden) te creëren. Hoe meer relaties hoe duidelijker de informatie is en een
grotere kans dat de individuen, waartussen de communicatie bestaat, overeenstemming vinden.
Communiceren is dus informatie uitwisselen door relaties te maken waardoor onduidelijkheden
tussen individuen reduceert.
In geschreven taal kan een enkele zin één relatie maken. Soms heb je meerdere zinnen nodig
om een complex idee uit te leggen maar dat complexe idee bestaat uit niet verder deelbare
zinnen. Een zin is dus is het meest elementaire deel om een relatie te maken.
Een zin kan uit verschillende frasen (woordgroepen) bestaan die weer uit woorden bestaat. Één
enkele zin kan samengesteld zijn uit verschillende zinsdelen.

\chapter{Categorieën}
Woorden en frasen kunnen beschreven worden in verschillende linguïstische categorieën. Je
kunt een woord bijvoorbeeld op lexicaal niveau beschrijven. Voorbeelden zijn: werkwoorden,
naamwoorden, lidwoorden, etc. Je noemt dit een woordsoort en beschrijft hiermee de functie
van het woord. Je kunt een woord ook grammaticaal beschrijven. Hiermee kun je bijvoorbeeld
aangeven in welke tijd of bij welk geslacht een woord hoort. Dit zijn de twee gangbare
categorieën die je op de basisschool krijgt en waar ik van verwacht dat je er mee bekend ben.
De gangbare categorieën vind ik niet altijd duidelijk. Daarom heb ik mijn eigen categorie
gecreëerd die ik hier zal verduidelijken. In het kort heb je woorden (of frasen) die een concept
benoemen. Deze concepten kunnen door relaties (relatie woorden) aan elkaar gekoppeld
worden. Een concept kan daarnaast gerefereerd worden (referentie woorden). Zowel relaties
als concepten kunnen verder gespecificeerd worden (specificeer woorden) en context aan
gegeven worden (context woorden).

\chapter{Woordsoorten}
Om een zin te maken heb je twee (lexicale) woordsoorten nodig, namelijk een zelfstandig
naamwoord (concept-woord) en een werkwoord (relatie-woord).
Mats kookt.
In deze zin is Mats het zelfstandig naamwoord en kookt het werkwoord. Dat een woord bij een
bepaalde lexicaal woordsoort hoort hangt af van de context.
Koken is leuk.
Hier is “koken” een zelfstandig naamwoord en “is” het werkwoord. Koken is hier dus geen
werkwoord. Laat je dus niet misleiden door de naam van een lexicaal woordsoort. Het woord
“is” is in deze zin het werkwoord. Proberen een betekenis uit de lexicaal woordsoort-naam te
halen heeft ook geen zin. (iets wat ik tevergeefs altijd heb gedaan.) Toch is het belangrijk om de
woordsoorten te herkennen binnen een zin. Door er achter te komen om wat voor woordsoorten
het gaat kun je de tekst beter begrijpen en spellen.

\chapter{De relatie}
De zin “Mats kookt” kun je ook grammaticaal beschrijven. Mats is het onderwerp en kookt
de persoonsvorm.
Hierbij is het onderwerp het woord waar het over gaat in de zin en de persoonsvorm een relatie.
In dit geval is niet nader toegelicht wat Mats kookt. Dit klink waarschijnlijk nog abstract maar zal
snel duidelijker worden. Vaak staat er in een zin een zogenaamd lijdend voorwerp.

Mats kookt pasta

Hierbij is pasta het lijdend voorwerp. In deze zin is er een duidelijke relatie (de persoonsvorm)
tussen Mats en Pasta.

\reuseMPgraphic{model}{
    lc="mats",
    r="kookt",
    rc="pasta",
    lcx="",rlcx="",rcx="",ls="",rls="",rs="",meta=""}

Dit is alles om een relatie tot stand te brengen. Je maakt een relatie tussen twee ongerelateerde
concepten. Alle “ruis” die een zin kan bevatten veranderd niet de relatie maar specificeert of
geeft context aan deze relatie.
Soms zie je een tekst —bijvoorbeeld op de voorkant van een magazine— waar geen relatie
gemaakt wordt, zoals: Mats op de camping. Hierbij is “op de camping” een frase en functioneert
als geheel om context te geven. (later meer over context). Dit is verwarrend want waarom is
“op” niet de relatie tussen Mats en camping. “Mats op de camping” verduidelijkt nog niet. De
relatie “op” heeft geen betekenis. Zou je schrijven “Mats is op de camping” dan bestaat er een
duidelijke relatie want Mats als concept wordt verduidelijkt. Hij ís namelijk op de camping.
Daarnaast is de zogenaamde pragmatiek van belang. Een individu kan iets anders bedoelen als
wat hij schrijft. Bijvoorbeeld de tekst: “waarom zijn bananen krom?” gaat niet om de relatie
tussen bananen en krom.

\chapter{Concepten}
Een concept is een woord waarbij —indien je dezelfde taal spreekt— je min of meer hetzelfde
beeld bij vormd. Bijvoorbeeld bij het concept pasta kunnen twee individuen een andere pasta in
gedachten hebben maar conceptueel is een spaghetti of een tagliatelle hetzelfde.
Het verschil in conceptualisatie is waar vaak spraakverwarring vandaan komt. Het is afhankelijk
van de kennis van een individu maar ook cultuur en tijd bepalen hoe een individu een woord
conceptualiseert.

\chapter{Specificeren}
Je wilt in een zin duidelijk maken dat het exact over het onderwerp of lijdend voorwerp gaat die
jij bedoeld. Daarom kun je ze nader specificeren.
Kleine Mats kookt verse pasta.
In deze zin wordt duidelijk dat het om kleine Mats gaat — niet de grote. Ook gaat het om verse
pasta.

\reuseMPgraphic{model}{
    lc="mats",
    r="kookt",
    rc="pasta",
    lcx="",rlcx="",rcx="",ls="kleine",rls="",rs="verse",meta=""}

Specificeer-woorden kun je ook gebruiken om andere specificeer-woorden te verduidelijken.
Bijvoorbeeld: “De hele kleine Mats kookt super verse pasta”.

\reuseMPgraphic{model}{
    lc="mats",
    r="kookt",
    rc="pasta",
    lcx="",rlcx="",rcx="",ls="hele kleine",rls="",rs="super verse",meta=""}

\chapter{Context}
Naast dat je woorden kunt specificeren kun je ze ook in context plaatsen.
Mijn Mats kookt veel pasta
Hierbij geef je met de woorden “mijn” en “veel” context aan. Het gaat om mijn Mats maar je zegt
niets over Mats zelf. Ook “veel” zegt niets over pasta (zoals verse) maar geeft aan dat Mats véél pasta kookt.

\reuseMPgraphic{model}{
    lc="mats",
    r="kookt",
    rc="pasta",
    lcx="mijn",rlcx="",rcx="veel",ls="",rls="",rs="",meta=""}

Anders dan met specificeer woorden kun je context-woorden ook toevoegen aan de relatie en
heeft ook alleen betekenis in de context van de relatie. Dat wil zeggen wanneer je pasta
specificeert als rode pasta dan heeft dat op zichzelf betekenis. Je visualiseert een concept
(pasta) maar hebt door rode een iets concreter beeld van de pasta. In tegenstelling tot
context-woorden als veel, gauw of geen geven alleen informatie door ze samen te zien met de
relatie. Kookt veel pasta, kookt gauw pasta, kookt geen pasta of, zoals het voorbeeld, mijn Mats
kookt.

\chapter{Zinsdelen}
Een zin kan uit zinsdelen bestaan. Een zinsdeel wordt altijd door een zogenaamd voegwoord
gemaakt.
Mats kookt pasta als hij jarig is.
Je kunt de zinsdelen vinden door ze te splitsen op het voegwoord, “als” in dit geval. Een
zinsdeel heeft altijd een onderwerp en persoonsvorm en maakt dus een relatie. De hoofdzin
(onafhankelijk van een ander zinsdeel), “Mats kookt pasta” is hier een goed voorbeeld van. De
bijzin “hij jarig is” is afhankelijk van de hoofdzin. Een zin kan uit meerdere zinsdelen bestaan en
de volgorde van een hoofdzin en bijzin is niet relevant. Ik visualiseer me een relatie tussen de
zinsdelen door een voegwoord zoals de persoonsvorm een relatie is tussen het onderwerp en
lijdend voorwerp.

Door deze manier van conceptualiseren wordt het ook duidelijk dat je met voegwoorden een
relatie maakt met verschillende ideeën. Overigens zal een taalpurist zeggen dat het woord jarig
geen lijdend voorwerp aanduidt maar het hulpwerkwoord “is” als persoonsvorm fungeert en
samen met het predicatief bijvoeglijk naamwoord het gezegde vormt … het zij zo. Toch is het
wel een interessant gegeven. Je zou ook kunnen zeggen de frase “is jaren” is een zogenoemde
unaire relatie met Mats. Net als het eerder genoemde voorbeeld “Mats kookt”. Hier zegt kookt
iets over Mats zelf.

\chapter{Frasen}
Frasen zijn woordgroepen die samen een woordsoort aanduiden. Anders dan een zinsdeel
bevat een frase niet een onderwerp en de persoonsvorm, het maakt dus geen relatie.
Mats, een meesterkok, kookt pasta
In deze zin, specificeer de frase “een meesterkok” Mats. Conceptueel kun je een frase dus ook
als woordsoort beschouwen.

\reuseMPgraphic{model}{
    lc="mats",
    r="kookt",
    rc="pasta",
    lcx="een meesterkok",rlcx="",rcx="",ls="",rls="",rs="",meta=""}

Je hebt zelfs woordsoorten die alleen iets betekenen als frase. Zo bestaat er voorzetsels die op
zichzelf geen betekenis hebben. Voorbeelden van voorzetsels zijn: voor, op boven, etc. maar in
een frase geven ze context
Mats kookt pasta op de camping

\reuseMPgraphic{model}{
    lc="mats",
    r="kookt",
    rc="pasta",
    lcx="",rlcx="",rcx="op de camping",ls="",rls="",rs="",meta=""}

Je zult merken dat veel zinnen voorzetsels bevatten. Door de frase als één woordsoort te zien
kom je sneller bij de relatie.

Op school heb je misschien geleerd dat je met een voorzetels een locatie aanduidt. (Een van de
weinige dingen die ik kon onthouden.) In mijn geval tekende de leraar een kar op het bord en
stonden daar voorzetsels omheen zoals: op de kar, naast de kar, in de kar, etc. Deze analogie
is alleen beperkt van toepassing. Ik kan me geen kar voorstellen bij voorzetsels als: ondanks,
gedurende, ongeacht. Wel kan ik de frase snel ontdekken.

Frasen moeten ook bij elkaar blijven staan. Ik visualiseer me een blok die als geheel verplaatst
moet worden om de betekenis gelijk te houden:

Op de camping kookt Mats pasta.
Mats kookt op de camping pasta.
Pasta kookt Mats op de camping.

\chapter{meta relaties}
De meesten woordsoorten specificeren of geven context aan andere woorden. We zagen al dat
een voegwoord zich anders gedraagt omdat deze relatie tussen zinnen vormt. Een
tussenwerpsel is zo’n ander woordsoort die zich op zinsniveau gedraagt. Het geeft context aan
een zin, meta-context dus. Meestal een oordeel (lees: relatie) die de spreker heeft over een zin.
Nouja, Mats kookt pasta.

\reuseMPgraphic{model}{
    lc="mats",
    r="kookt",
    rc="pasta",
    lcx="",rlcx="",rcx="",ls="",rls="",rs="",meta="nouja"}

Vergelijkbaar met meta-context bestaan er ook relaties die op een volledige zin duiden in plaats
van een concept. Een voorbeeld is: hij zegt “Mats kookt pasta”. Hierbij is de relatie zegt
verbonden met de zin Mats kookt pasta.

\reuseMPgraphic{model}{
    lc="mats",
    r="kookt",
    rc="pasta",
    lcx="",rlcx="",rcx="",ls="",rls="",rs="",meta="hij zegt"}

\chapter{Herkenning}
Met dit model is het herkennen van woorden een stuk eenvoudiger voor mij omdat je de
afhankelijkheden terug kan brengen naar een simpel model. Aan de hand van een aantal
voorbeelden laat ik zien hoe je dit model kan gebruiken om je vocabulaire te vergroten. Om dit
beter uit te leggen geef ik eerst een korte introductie tot woordvorming.

\section{Woordvorming}
Woorden kunnen zelf opgesplitst worden in delen, zogenoemde morfemen die betekenis geven
aan een woord. Dit is praktisch omdat je anders voor elk concept of relatie een volledig nieuw
woord nodig hebt. Daarnaast worden de woorden vaak opgebouwd volgens een vooraf bepaald
schema. Vooral dit laatste was voor mij een nieuw inzicht\footnote{https://linguistics.as.uky.edu/video/defaults-morphological-theory-workshop-inheritance-and-construction-morphology} waardoor spelling en herkenning van
woorden een stuk eenvoudiger werd. Ik heb woorden altijd gezien als op zichzelf staande
betekenissen. De woorden als aartsbisschop en aartsvijand hadden voor mij geen verband en
moest ik individueel onthouden. Nu weet ik dat ze in een schema passen van het morfeem
aarts-. Het schema ziet er als volgt uit:
\blank

[aarts-, [x]z1]z \Longleftrightarrow [hoogste range x1]

\blank
Hier staat zoiets als alle woorden met een prefix aarts- voor een woord (als variabele x) van het
soort zelfstandig naamwoord (subscript z1) heeft de betekenis hoogste rang van variabel x en is
zelf ook een zelfstandig naamwoord. Wellicht heb je nog nooit van het woord aartslui gehoord
maar weet nu wel de betekenis. Mijn verwachting is dat je dit niet direct begrijpt maar door een
aantal voorbeelden ga je het patroon herkennen.
\blank
Wat je hier eigenlijk doet is een concept specificeren zonder een specificeer woord toe te voegen.
\reuseMPgraphic{model}{
    lc="mats",
    r="kookt",
    rc="aartspasta",
    lcx="",rlcx="",rcx="",ls="",rls="",rs="",meta=""}

Nu heb ik geen idee wat ik me bij aartspasta moet voorstellen. (iets van paste met de beste 
kwaliteit of iets dergelijks). Maar het gaat hiet om het voorbeeld. Het concept blijft staan.
\blank

[[x]w1, -baar]bn \Longleftrightarrow  [kunnen ondergaan x1]

\blank
In dit schema wordt een werkwoord (w1) omgezet naar een bijvoeglijk naamwoord (bn) met de
betekenis kunnen ondergaan zoals breekbaar en schaalbaar. M.a.w. je veranderd een relatiewoord
naar een specificeerwoord.
\blank
Zie wat er gebeurd met de volgende zin waarbij ik de relatie omzet.
\blank
\reuseMPgraphic{model}{
    lc="mats",
    r="kookt",
    rc="pasta",
    lcx="",rlcx="",rcx="",ls="",rls="",rs="",meta=""}
\blank
de relatie kook zetten we om en geven aan dat Mats pasta heeft. 
\blank
\reuseMPgraphic{model}{
    lc="mats",
    r="heeft",
    rc="pasta",
    lcx="",rlcx="",rcx="",ls="",rls="",rs="kookbare",meta=""}
\blank
Nu weten we dat het om kookbare pasta gaat.
\blank

[[x]bn1, -heid]z \Longleftrightarrow  [verkeren in x1]

\blank
In dit schema wordt een bijvoeglijk naamwoord omgezet naar een zelfstandig naamwoord met
de betekenis, verkeren in, zoals beleefdheid en absoluutheid.
\blank

[[x]z1, -achtig]z \Longleftrightarrow  [in rijke mate x1]

\blank
In dit schema blijft het lexicale woordsoort gelijk maar geeft het de betekenis van in rijke mate
zoals bergachtig of rimpelachtig.
Je kunt eenvoudig nieuwe woorden creëren met deze schema's waardoor je vocabulaire
ontzettend groeit. Een voorbeeld van een correct nederland onbestaand woord op basis van de
laatste schema's is: fietsbaarheidachtig.
Er zijn twee verschillende vormen van woordvorming. De hierboven genoemde voorbeelden
vallen alle onder zogenoemd derivatie waarbij nieuwe woorden worden gevormd. Naast
derivatie bestaat er flexie. Hierbij worden woorden veranderd op basis van kenmerk zoals tijd of
geslacht. De schema’s werken voor flexie minder goed. Wel gebruik ik een eenvoudige variant
om de regels uit te drukken welke we in het hoofdstuk spelling terug zullen zien.

\chapter{Spelling}
Taal is organisch. Dwz. er zijn niet altijd duidelijke regels waarom dingen geschreven worden.
Soms komt dat door de herkomst van woorden (etymologie) toch zijn er duidelijke regels die we
kunnen toepassen. De vraag is alleen wanneer je de juiste regels kunt toepassen. Dit hoofdstuk
laat zien hoe je het model toepast om de juiste spelling toe te passen.

\section{Tijd}
In een zin is tijd vaak een belangrijk onderdeel om je idee over te brengen. Nu was het voor mij
altijd ontzettend moeilijk om überhaupt te herkennen of een zin in de tegenwoordige tijd of
verleden tijd stond laat staan of deze al dan niet voltooid was.

\section{Tegenwoordige tijd}
Nu weet ik dat de tijd aangegeven wordt in de relatie. Bijvoorbeeld in de zin: Mats kookt pasta.
Bevat de relatie “kookt” de tijd. De regel die daar bij hoord is: je voegt altijd een “t” toe tenzij het
1ste persoon (dwz. “ik”) is of voor een vragend voornaamwoord staat (vind jij? Niet: vindt jij?).
Dus ik brand en hij brandt.
\blank

[[x]v, -t]v als x geen 1ste persoon, gebiedende wijs of vragend voornaamwoord is

\blank
\section{Verleden tijd}
Lastiger is de verleden tijd zoals Mats kookte pasta. Je voegt hier “de” of “te” wanneer het
enkelvoud is– bij meervoud plaats je nog een -n erachter. Ook is het lastig wanneer je nu een t
of d schrijft gelukkig weet je nu wanneer je ‘t kofschip moet gebruiken.
\blank

[[x]v , -te(n)/-de(n)]v

\blank
\section{Voltooid}
Elke tijd heeft nog een micro tijd. Hier kun je nog iets aan het doen zijn (onvoltooid) of je heb het
net gedaan (voltooid). Bijvoorbeeld de zin: “Mats heeft Pasta gekookt”. Door weer naar de
relatie te kijken kun je vrij snel zien dat het morfeem “ge-” is toegevoegd. Deze relatie heeft
altijd een hulpwerkwoord en kan je daarom ook in de relatie plaatsen. In dit voorbeeld “heeft
gekookt”. Heeft is tegenwoordige tijd. Dat wil zeggen dat Mats, {\em nu net}, zijn pasta heeft gekookt.
Zou je willen uitdrukken dat Mats gisteren (dus in het verleden) pasta heeft gekookt dan
verander je het hulpwerkwoord in had. Dus Mats had pasta gekookt.
Het fijne van voltooide werkwoorden is dat de spelling altijd hetzelfde blijft. Je kunt het volgende
schema gebruiken
\blank

[ge- [x]v -t/-d]

\blank
\section{Specificeren}
Wanneer je een concept specificeert gaat het altijd om een bijvoeglijk naamwoord en eindigt (op
een aantal uitzonderingen na) op een -e. Bijvoorbeeld: Grote Mats kookt lekkere Pasta. En niet:
groten Mats kookt lekker pasta. Je kunt een bijvoeglijk naamwoord ook zelf specificeren. Dit zijn
altijd bijwoorden. Deze bijwoorden vind ik zelf lastiger te identificeren en blijven meestal over
nadat ik de andere woorden heb ontleed. Het gaat bijvoorbeeld om woorden die de
hoedanigheid aangeven. Bijvoorbeeld: Mats kookt bijzonder lekkere Pasta. Hierbij specificeer je
het bijvoeglijk naamwoord lekker door bijzonder.

\section{De komma}
Voor mij was het altijd lastig om te bepalen waar de komma komt. (nu is dit ook niet heel strikt
maar soms wel noodzakelijk). Door het verschil tussen specificeren en context te weten is dit nu
eenvoudiger. Bijvoorbeeld de zin “Mats kookt uitsluitend, verse pasta” betekent dat Mats
uitsluitend (verse) pasta kookt maar niets anders. Je geeft dus context aan. Ter vergelijking met
de zin: “Mats kookt uitsluitend verse pasta” zegt iets over verse. Dus Mats kookt met uitsluitend
verse pasta maar misschien maakt hij er nog wel een salade bij.


\chapter{De praktijk}
We hebben nu vooral eenvoudige zinnen gebruikt als “Mats kookt pasta”. Dit hoofdstuk laat zien
hoe je het model in de praktijk gebruikt. Door een aantal voorbeelden te gebruiken
(krantenartikelen, magazines, et cetera) laat ik zien hoe ik deze lees en interpreteer volgens het model.
Het model helpt me maar je kunt ook zien welke worsteling ik nog steeds heb.  

\section{spelling van werkwoorden}

De volgende zinnen komen uit een schoolopdracht van mijn tienjarige dochter. 
De opdracht bestaat uit het juist invullen van het missende werkwoord. Het hele werkwoord wordt gegeven.
Als kind kwam ik hier echt niet uit en zie ook dat mijn dochter er moeite mee heeft. Niet omdat ze dyslectisch is 
maar omdat de regels lastig uit te leggen zijn.

Nu weet ik dat ik me alleen maar op de relatiewoorden hoef te richten. Dit woord is al gegeven dus is het alleen
nog maar de verbinding zoeken en de regels toepassen. 
\blank
\zin Vorig jaar {em trachtte} een inbreker ons huis binnen te dringen

\reuseMPgraphic{model}{
    lc="inbreker",
    r="trachtte",
    rc="binnen te dringen",
    lcx="",rlcx="vorig jaar",rcx="ons huis",ls="",rls="",rs="",meta=""}

Het werkwoord trachten wordt hier als trachtte gespeld omdat inbreker enkelfout is en de tracht de stam (stam + te).
verwarrend aan deze zin is wellicht het woord dringen. Dit is een zogenaamde onbepaalde wijs en kan je op het verkeerde
been zetten. Ik zou dit als relatiewoord gezien kunnen hebben. Dit zou ook zo kunnen zijn als de zin iets anders
neergezet was. Het volgende voorbeeld laat dit zien.

Dit jaar dringen inbrekers ons huis binnen. 

\reuseMPgraphic{model}{
    lc="inbrekers",
    r="dringen",
    rc="binnen",
    lcx="",rlcx="dit jaar",rcx="ons huis",ls="",rls="",rs="",meta=""}
.

\zin Lenie {\em stond} meer dan een kwartier op de bus te wachten

\reuseMPgraphic{model}{
    lc="lenie",
    r="stond",
    rc="te wachten",
    lcx="",rlcx="meer dan een kwartier",rcx="op de bus",ls="",rls="",rs="",meta=""}

Weer zien we in deze zin een onbepaalde wijs (komt door het woordje 'te') maar een andere hoofdbreker is het onregelmatige
werkwoord. In dit geval, het werkwoord stond. Nu heb ik niet zoveel moeite met dit voorbeeld maar loop hier wel eens tegenaan. denk bijvoorbeeld
aan stofzuigen. Is het nu stofzuigde of stofzoog? In deze situaties ga ik naar de website van de Genootschap Onze Taal 
\footnote{https://onzetaal.nl/taaladvies/zoek-spelling}. Overigens had hier ook 'staat' kunnen staan.


\zin Heb je gezien, wie daar in de snackbar {\em bediend}?

Oefening 1, zin 3 hoe moeilijk wil je het maken? Het lastige van deze zin is dat het eigenlijk twee zinnen zijn die met een
voorzetsel aan elkaar zijn verbonden. Daarnaast is het een vraag en vragen beginnen altijd met een werkwoord. In het schema leest 
het dus minder lekker. Om met de eerste zin te beginnen. De relatie bestaat uit het hulpwerkwoord heb. Deze verbind het 
verwijzend voornaamwoord 'je' met het voltooid deelwoord 'gezien' zoals het volgende schema laat zien.


\reuseMPgraphic{model}{
    lc="je",
    r="heb",
    rc="gezien",
    lcx="",rlcx="",rcx="",ls="",rls="",rs="",meta=""}


De tweede zin wordt verbonden met het voorzetsel daar. Het werkwoord bediend verbind het verwijzend voornaamwoord wie. Het is
bediend en niet bediende omdat het om de voltooid tegenwoordige tijd gaat en kun je zien door de relatie heb. het is niet had.
n omdat het voltooid is schrijf je geen t achter bediend. 

\reuseMPgraphic{model}{
    lc="wie",
    r="bediend",
    rc="(gasten)",
    lcx="",rlcx="in de snackbar",rcx="",ls="",rls="",rs="de",meta=""}

wat je aan dit voorbeeld mooi kunt zien hoe taal werkt. Eigenlijk zou de zin moeten zijn. Heb je gezien, wie daar 
de gasten, in de snackbar, bediend. De gasten is weggelaten en wordt verondersteld dat dit duidelijk is of niet noodzakelijk.
\break
\zin Vroeger {\em werkten} men van 's morgens vroeg tot 's avonds laat.

\reuseMPgraphic{model}{
    lc="men",
    r="werkten",
    rc="van 's morgens vroeg tot 's avonds laat",
    lcx="",rlcx="vroeger",rcx="",ls="",rls="",rs="",meta=""}


\chapter{Achtergrond}
Dat ik een taalachterstand had wist ik maar werd pas echt zichtbaar toen ik mijn vrouw leerde
kennen. Zij is iemand die heel erg talig is.
Wanneer we samen door een onbekende stad wandelen weet zij precies alle straten te
benoemen waar we zijn geweest. Waar we ons op dat moment bevinden heeft ze meestal geen
flauw benul van.
Ze kan niet begrijpen, wanneer er een overduidelijk bord staat met de tekst “gesloten”, ik dit niet
heb gelezen. Wel kan ik vertellen welke typografie er bij het bord gebruikt is en of dit in
overeenstemming is met het gevoel van de winkel.
Hetzelfde zie ik bij mijn kinderen. Mijn dochter Femke (6) weet precies uit te leggen hoe een
bestuiving werkt. “De Bij pakt het nectar van een bloem en neemt het dan mee naar een
volgende bloem om het te bestuiven”. Woorden als nectar en bestuiven zijn voor mij heel
concreet maar zij gebruikt het zonder erbij na te denken. Ik zou iets antwoorden als: “je hebt
plantjes en beestjes verplaatsen stofjes”. Natuurlijk weet ik dat planten geen bloemen zijn en
kan ik het best beter formuleren maar daar moet ik echt goed over nadenken.
Ook mijn zoon Mats heeft het talent van zijn moeder. Hij is 3 jaar maar weet al abstracte
woorden te gebruiken. Je hoort hem oefenenen en zit bijna altijd goed. “Ik wil alsnog een
broodje”. “Eigenlijk weet ik het niet. Weet jij dat”.
Door deze verschillen is het mij duidelijk geworden dat ik nooit een talenwonder zal worden. Ik
lees niets waardoor ik ook niets leer, ik heb geen geheugen om concrete woorden op te slaan
en ben niet bezig om abstracte woorden te gebruiken. Desalniettemin kan ik mijzelf verbeteren.

\section{Ik hou van jou}
Hoe zeg je dat je van iemand houd? De eerste paar jaar (niet overdreven) dat ik dit naar mijn
vriendin, mijn huidige vrouw, wilde sturen heb ik dit steeds op moeten zoeken. Ik had geen idee.
Is het “ik houd van jou” (formeel) of ”ik hou van jouw (zus?)”. Of is het toch “ik hou van je (en
ook van je zus btw)”. Met taal kun je zoveel uitdrukken maar nog meer verknoeien. Lieverd, ik
hou van jou!

\stoptext:

\end{document}
